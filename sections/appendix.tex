\section{Appendix}

\subsection{Calculations for Pumped Storage with Heavy Piston Design}
As mentioned above, a company called Gravity Power is developing an energy storage design which involves a heavy piston suspended in a vertical shaft filled with water. The system employs a closed loop of water with a pump and a power generator. The system can cycle water in both directions. Energy is expended to lift the piston by pumping water under it. Energy can be recaptured by allowing the weight of the piston to push the water back up through a turbine. When the piston falls, the water gathers in the space above the piston.

With these assumptions about the piston design, our calculations below demonstrate that this design should be less efficient than underground pumped hydro energy storage.

Let's begin with the basic equation for gravitational potential energy where E is energy, m is mass, g is the gravitational constant, and h is height.
\[ E = mgh \]
We replace mass with volume times density ($\rho$) where volume is represented by the height (h) times area (a).
\[ E = (h*a*\rho)*g*h \]

The piston design system includes two densities. We will define the density of the rock piston as ($\rho_p$) and the density of the water as ($\rho_w$).

The piston and water volumes both occupy the entire cross section area of the shaft. The potential energy of each piece depends on each of their heights and the distance (d) that each travels. We must remember that when the piston drops, the water is lifted above the piston. So the water's potential energy must be subtracted from the piston's. Using subscripts $_p$ for piston and $_w$ for water, we have:
\[ E_{net} = (h_p*a*\rho_p)*g*d_p - (h_w*a*\rho_w)*g*d_w  \]

We introduce f as the fraction of the shaft height which the piston will occupy. We can then translate the heights of the piston ($h_p$) and height traveled by the piston ($d_p$) as functions of f.
\[ h_p = hf \ , \ d_p = h-hf \]

Because the water occupies the rest of the volume around the piston, we can also define the height of the water and the height traveled by the water as:
\[ h_w = h-hf\ , \ d_w = hf \]

These are naturally the inverse values as the piston, because each element must travel the distance of the other's height to balance the volume of the closed system.

Now let's revisit our Energy equation and begin substitution.
\[ E_{net} = (h_p*a*\rho_p)*g*d_p - (h_w*a*\rho_w)*g*d_w  \]

Refactoring, we have:
\[ E_{net} = ag(\rho_p(hf)(h-hf) - \rho_w(h-hf)(hf)) \]
We note the common factor of $(hf)(h-hf)$, and refactor again.
\[ E_{net} = ag((hf)(h-hf)(\rho_p - \rho_w)) \]

Since rock is about 2.5 denser than water we have
\[ \rho_p = 2.5\rho_w \]
So
\[ E_{net} = ag((hf)(h-hf)(2.5\rho_w - \rho_w)) \]
Simplifying
\[ E_{net} = ag(fh^2 - h^2f^2)(1.5\rho_w) \]
Refactoring
\[ E_{net} = 1.5ag\rho_wh^2(f - f^2) \]

Inspecting this equation, we see that the maximal f value is .5. So we conclude that the piston should be half the height of the shaft. This gives us:
\[ E_{net} = 1.5ag\rho_wh^2(.5 - (.5)^2) \]
or
\[ E_{net} = 0.375ag\rho_wh^2 \]

Now let's compare this result to the potential Energy $E_{UPHS}$ of the shaft filled with just water and no rock piston. We note that the water will, on average, only rise half the height of the shaft as it empties to ground level at the top of the shaft. We use our equation above but with new values for water using a subscript u. So our water has a height $h_u$ and distance traveled $d_u$.

\[ E_{UPHS} = (h_u*a*\rho_w)*g*d_u \]
The water is the full height of the column and will travel half the height on average.
\[ h_u = h \ , \ d_u = .5h \]
So we have
\[ E_{UPHS} = ag\rho_w(h)(.5h) \]
or
\[ E_{UPHS} = .5ag\rho_wh^2 \]
We can see that this is an improvement over $E_{net}$ for the piston design. With common values for $ag\rho_wh^2$, the improvement of UPHS over this piston design appears to be
\[ (0.5 / 0.375) - 1 = 0.\bar3 \]

In other words, UPHS should give an improvement of about 33\% over this piston design. With this result, the piston design does not seem to be compelling alternative to UPHS. We do concede however that this piston design eliminates the need for an upper reservoir. So perhaps the idea could still prove valuable for some locations where an upper reservoir is not possible.

\subsection{Calculation of New York City Daily Energy Usage}
New York State's Total Annual Energy usage in 2017 was about 905 trillion Btu\cite{NewYorkStateEnergyProfile}. This is equivalent to 265,229,319 MWh.

According to the New York Times, “nearly 60 percent of the state’s electricity is consumed in the New York City area.” \cite{HowNewYorkCityGetsItsElectricity} This number is validated by the eia.gov report which says “two-thirds of the state's power demand is in the New York City and Long Island region” \cite{NewYorkStateEnergyProfile}. So we will use 60\% and consider this the New York City Area not including Long Island. This gives us
\[ 265,229,319MWh * .6  / 365 \approx 435,993MWh \]

The New York City area consumes approximately 435,993MWh of power daily, or
\[ 435,993MWh / 24 \approx 18,166MWh \]
18,166MWh per hour

New York City has pledged to have 500 MW of storage available by 2025. But as we can see
\[ 500MW * 1 hour / 18,166MWh \approx 0.0275 \approx 2.8\% \]

This will only yield about 2.8\% of NYC's total power consumption in storage energy.


\subsection{Calculation of Vertical Shaft Volume and Cost}
In the PNL report, it was estimated that the lower reservoir of a UPHS installation might have a volume of about 6,012,900 $m^3$. The volume of the vertical shafts is not given, but the dimensions are. The PNL report mentions there could be 4 shafts with diameters of 5.8 m ranging in depth from 1525 to 1677 m. \cite{UndergroundPumpedHydroelectricStorage} We'll use 1600 m as an average value. This gives us an estimated volume for each shaft:
\[ \pi r^2 * h = \pi (5.8 / 2)^2 * 1600 \approx 42,273 m^3\]

For all four shafts, this gives us 169,092 $m^3$
\[ 42,273 m^3 * 4 = 169,092 m^3\]

We note that according to these estimates, the mined vertical volume is about 3\% of the total volume.
\[ 169,092 m^3 / (6,012,900 m^3 + 169,092 m^3) \approx 0.03 \]

The PNL report estimates that the lower reservoir of a UPHS plant would likely represent about 30\% of the overall project cost. \cite{UndergroundPumpedHydroelectricStorage} Given that the vertical shafts represent only 3\% of the overall mined volume, we consider it negligible for the purposes of our rough estimations. We'll estimate that all the digging still represents about 30\% of the overall cost.


\subsection{Calculation of LCOE for UPHS}
According to the PNL report, the total direct cost of a UPHS plant built in 1983 would be estimated at $\$500 * 10^6$ for a 1000-MW plant in 1983 dollars.\cite{UndergroundPumpedHydroelectricStorage} (about \$1.3 billion in today's dollars. \cite{CPIInflationCalculator}). Note that this works out to \$1,300/kW.

We will base our LCOE calculation on this UPHS construction cost of \$1,300,000,000 for a 1000-MW plant.

Our calculation will use NREL's formula for Simple Levelized Cost of Energy:

\begin{displayquote}
sLCOE = \{(overnight capital cost * capital recovery factor + fixed O\&M cost) / (8760 * capacity factor)\} + (fuel cost * heat rate) + variable O\&M cost. \cite{SimpleLevelizedCostOfEnergyCalculator}
\end{displayquote}

We'll remove variable O\&M cost and fuel cost which are not relevant for pumped hydro. We assume a capacity factor of 30\%. We define overnight capital cost as $C_i$ dollars per installed kilowatt (\$/kW). We assume a lifetime of 40 years. We'll assume the fixed O\&M costs are (2.5\% * $C_i$) dollars per kilowatt-year (\$/kW-yr). We define capital recovery factor as CRF, which per NREL, is defined as the following where i is the interest rate and n is the number of years.
\[ \displaystyle CRF={\frac {i(1+i)^{n}}{(1+i)^{n}-1}} \]

This gives us:
\[ LCOE = ((C_i * CRF) + (.025 * C_i)) / (8760 * .3) \]

We will assume a conservative rate of 10\% interest. At 10\% interest over 40 years, our CRF is about 0.1023
\[ \displaystyle CRF={\frac {0.1(1+0.1)^{40}}{(1+0.1)^{40}-1}} = 0.10225941441 \]

Which gives us:
\[ LCOE = (0.1023C_i + .025C_i) / 2628 \]
which reduces to
\[ LCOE \approx 0.00004844 C_i \]

Using our initial cost above, we determine our value of $C_i$ in dollars per installed kilowatt. Our value is for a 1,000MW capacity plant (1,000,000kW).

\[ C_i = \$1,300,000,000 / 1,000,000kW = \$1300/kW\]

This gives us a LCOE for UPHS of about \$0.063/kWh, or \cent 6.3/kWh
\[ LCOE \approx \$(0.00004844 * 1300)/kWh \approx \$0.063/kWh \]

We note that this estimate may be simplified and idealized. We also note that it is about 2.4 times cheaper than the lowest contemporary estimate we could find for pumped hydro storage.
\[ 0.15 / 0.063 = \approx 2.4\]

This discrepancy could be due to a number of factors. One factor could be that contemporary estimates of pumped hydro projects may average out the costs of actual built projects. Some of these built projects may have suffered from cost overruns, or they may have been built on less suitable locations with smaller overall capacities which would have lost economies of scale. Underground pumped storage could arguably avoid such risks as it is not as site dependent. See \textit{Comparison of researched LCOE for PHS, UPHS, and Li-ion Batteries}.

\subsection{Calculation of LCOE for UPHS (Optimistic)}
Here we will re-calculate the above LCOE using more optimistic predictions.

The original estimated cost of UPHS construction was \$1,300,000,000. We speculate that this cost could be feasibly lowered by 25\% if a UPHS plant was built onto an existing dam. This brings the cost down to \$975,000,000.

We divide this into two costs: The cost of tunneling and everything else.

Tunneling was estimated to be 30\% of \$1.3 billion, or \$390 Million
\[ \$1,300,000,000 * .3 = \$390,000,000\]

Subtracting this out from the base cost, we see that non-tunnel costs are \$585 Million.
\[ \$975,000,000 - \$390,000,000 = \$585,000,000\]

As discussed previously, we presume that tunneling costs could be 5 times cheaper thanks to technological efficiencies. This would reduce tunnel costs to \$78 Million.
\[ \$390,000,000 / 5 = \$78,000,000\]

Adding these costs back together we get \$663 Million.
\[ \$78,000,000 + \$585,000,000 = \$663,000,000 \]

We take note that this price is just about half of our original estimated construction cost.
\[  \$663,000,000  / \$1,300,000,000 = .51 \approx .5 \]

To determine our LCOE, we'll use the same NREL formula mentioned above. We'll again remove variable O\&M cost and fuel cost. We'll assume a lifetime of 40 years which is conservative for PHS. All other assumptions are the same including a capacity factor of 30\%.

As discussed above, the current 30 Year US Treasury Rate is very low at 2.26\%. We will use an optimistic value of 3\% interest. At 3\% interest over 40 years, our CRF is about 0.0433.
\[ \displaystyle CRF={\frac {0.03(1+0.03)^{40}}{(1+0.03)^{40}-1}} = 0.04326237789 \]

Which gives us:
\[ LCOE = (0.0433C_i + .025C_i) / 2628 \]
which reduces to
\[ LCOE \approx 0.00002599 C_i \]

Using our initial cost above, we determine our value of $C_i$ in dollars per installed kilowatt. Our value is for a 1,000MW capacity plant (1,000,000kW).

\[ C_i = \$663,000,000 / 1,000,000kW = \$663/kW\]

This gives us a LCOE of about \$0.017/kWh, or \$17/MWh, or \cent 1.7/kWh.
\[ LCOE \approx \$(0.00002599 * 663)/kWh \approx \$0.017/kWh \]


\subsection{Calculation of LCOE for Li-ion}
Li-ion batteries currently have a lifespan of only 5-10 years. \cite{The3BillionPlanToTurnHooverDamIntoAGiantBattery} In ideal conditions they might have a lifespan of 7-10 years. \cite{LifePredictionModelForLiIonBattery}

Li-ion batteries have a levelized cost of \$187/MWh today. This cost is expected to more than halve in cost by 2040, reducing them to an LCOE of \$67/MWh. Assuming a linear decline in cost, the average cost will be \$127/MWh over the next twenty years. \cite{NewEnergyOutlook2019Report}
\[ (\$187/MWh + \$67/MWh) / 2 =  \$127/MWh \]

Let's make an assumption that Li-ion prices will halve again by 2060. That would bring down the LCOE to about \$34/MWh. Again, assuming a linear decline in cost, that would give us an average cost of \$110.5/MWh over the next forty years.
\[ (\$187/MWh + \$34/MWh) / 2 =  \$110.5/MWh \]

We note that over the lifetime of a Li-ion battery, the capacity declines steadily in a nearly-linear fashion from about 75\% to about 40\% of their capacity. \cite{LifePredictionModelForLiIonBattery} We're not certain whether this battery capacity decline is already averaged out in estimations of LCOE, so we will ignore this. But we note that if this has not been properly taken into account, this could strengthen UPHS's advantage over Li-ion.

We will assume a ten year lifespan for Li-ion batteries. As noted above, this could be lower in practice - possibly closer to 5-7 years. As noted below, it could also be larger due to technological advancements. But this has yet to be fully validated.

With these assumptions, we conclude that over forty years, if a Li-ion energy storage facility had to replace their batteries once a decade, their cost over forty years would be \$332/MWh.
\[ \$110.5/MWh * 3 \approx \$332/MWh \]

\subsection{Calculation of LCOE for Li-ion (Optimistic, dependent on Technology Advances)}
As mentioned above, Li-ion batteries currently have a lifespan of only 5-10 years (see \textit{Calculation of LCOE for Li-ion}).

New research has emerged however which suggests that lifespans of lithium ion batteries may be stretched to 20 years. \cite{ExcellentLithiumIonCellChemistry} This research was done by Jeff Dahn and his lab, who are doing battery research for Tesla, Inc. The gains are attributed to a design which includes a next-generation “single crystal” NMC cathode and a new advanced electrolyte.

Using this new optimistic figure, we can conclude that over forty years, a Li-ion energy storage facility would only need to replace their batteries once. So their cost over forty years would be \$221/MWh.
\[ \$110.5/MWh * 2  = \$221/MWh \]

We note that this figure is now within the ranges of research cited below (see \textit{Comparison of researched LCOE for PHS, UPHS, and Li-ion Batteries}).


\subsection{Comparison of researched LCOE for PHS, UPHS, and Li-ion Batteries}
In the sections above, we calculated a simplified LCOE for UPHS. We determined that the LCOE was \$0.063/kWh based on research from the 1980s (See \textit{Calculation of LCOE for UPHS}). We acknowledge that this estimate may be a simplified calculation which may ignore certain factors. Although we could find little to no contemporary research about UPHS pricing, in this section we compare our findings with more contemporary research for PHS pricing in general.

We also calculated a LCOE for Li-ion energy storage projected over 40 years. We determined that it could be somewhere between \$221/MWh and \$332/MWh. This correlates with outside research as we will show below.

The Lazard Energy Storage Report is one of the most trusted industry sources for price analysis. In 2016, the report determined the following for unsubsidized levelized costs of storage at grid-scale (transmission system)\cite{LazardsLevelizedCostOfStorageAnalysis2016Version2}:

\begin{tabular}{ ll }
Pumped Storage: & \$152/MWh - \$198/MWh  (about \$0.15/kWh - \$0.2/kWh) \\
Lithium-Ion: & \$267/MWh - \$561/MWh  (about \$0.25/kWh - \$0.55/kWh) \\
\end{tabular}

As shown, in 2016, pumped storage was estimated to about 6 times cheaper than lithium-ion on average for grid-scale applications.
\[ ((561-267)/2) / ((198-152) / 2) \approx 6.4 \]

Examining only the low estimates, pumped storage was estimated to be almost half the cost of lithium-ion for grid-scale applications.
\[ 267 / 152 \approx 1.8 \]

Unfortunately, Lazard began omitting PHS in 2017. Despite the very low LCOS (levelized cost of storage), it seems that the Lazard report decided to lump PHS into technologies with “limited current or future commercial deployment expectations.” \cite{LazardsLevelizedCostOfStorageAnalysis2018Version4} As discussed above, this is likely due to concerns over steep upfront costs and the diminishing availability of sites for new dam construction. However, Lazard is now expected to reincorporate PHS into the upcoming 2019 version of their report, according to researchers working with the San Diego County Water Authority. \cite{PumpedEnergyStorageVitalToCalifornia}

In their 2019 research paper, the same San Diego group calculated their own LCOS estimates for both PHS and Li-ion: \cite{PumpedEnergyStorageVitalToCalifornia}

\begin{tabular}{ ll }
Pumped Storage: & \$177/MWh  (about \$0.18/kWh) \\
Lithium-Ion: & \$218/MWh - \$285/MWh  (about \$0.22/kWh - \$0.29/kWh) \\
\end{tabular}

Summarizing their findings, the San Diego group concludes that even with drastic technology improvements for Li-ion, over a 40 year time frame “[Li-ion] batteries will remain overall more expensive than pumped storage — possibly 50\% more expensive than pumped storage.” \cite{PumpedEnergyStorageVitalToCalifornia} This is a notably different conclusion from some competing research which concludes that Li-ion will become cheaper than pumped storage.\cite{ProjectingTheFutureLevelizedCostOfElectricityStorageTechnologies} But the San Diego group believes that their research is more complete. They argue that the estimated lifetime cost of Li-ion increases after considering the nuances of long term financing. They explain: “Compared with pumped storage, the capital cost for these [Li-ion] projects is lower (\$285,000/MWh to \$452,000/MWh), but that cost must be paid back more rapidly because the lifetime of these [Li-ion] projects is shorter. The overall effect is to make batteries more expensive than pumped storage when those costs are levelized over the relevant time period.” \cite{PumpedEnergyStorageVitalToCalifornia}

Further, the San Diego group showed that PHS projects could continue to operate beyond 50 years with relatively little maintenance cost. Building on the existing infrastructure, the second 50 years of a PHS project has an extremely low LCOS of only \$58/MWh  (about \$0.06/kWh).


Summarizing this range of cost estimates:
We note that our LCOS estimate for UPHS based on 1980s data is significantly lower at \$63/MWh than contemporary estimates which show costs closer to \$150/MWh or \$175/MWh. Further research is needed to analyze this price gap and determine which details account for the difference (note that inflation has already been considered).

We acknowledge that these various estimates for UPHS have a wide range. But this should perhaps be expected considering that 1: Large UPHS projects have never been constructed before, and 2: Even contemporary research shows a fair amount of variation since it draws data from many disparate projects, each with geographic and financial differences.

Based on this range, our conservative estimate for UPHS will be \$177/MWh. Our optimistic LCOS baseline for UPHS will be \$63/MWh, which is based on our analysis of the 1980s PNL research. Finally, our very-optimistic LCOS will be \$17/MWh based on optimistic circumstantial improvements in favor of UPHS. See \textit{Calculation of LCOE for UPHS (Optimistic)}.


Based on the ranges of LCOE estimates for Li-ion batteries, we will use \$220/MWh as a feasible, best-case scenario for Li-ion. And we'll use \$285/MWh as a still feasible conservative cost for Li-ion. It's important to note that this is not a worst case scenario by any means. In fact, it is arguably still optimistic as it assumes that Li-ion technology will drastically improve in the coming decades.

\noindent\textbf{Summarizing all of the above:}
\begin{itemize}
  \item PHS LCOS estimates in 2016 were about 2-6 times cheaper than Li-ion.
  \item PHS LCOS estimates today, are still expected to be cheaper than Li-ion over a 40 year time frame.
  \item PHS facilities can extend their lifespan for 100 years or more with very low additional cost.
  \item UPHS cost estimates are projected to be about the same cost or cheaper than PHS in general.
\end{itemize}

We conclude that UPHS facilities built in the near future will be cheaper, maybe even far cheaper, than Li-ion facilities over the next 40-100 years.

\noindent\textbf{40-year LCOE values for UPHS and Li-ion}

\begin{tabular}{ llll }
UPHS (conservative): & \$177/MWh & Li-ion (conservative): & \$285/MWh \\
UPHS (optimistic): & \$63/MWh & Li-ion (optimistic): & \$220/MWh \\
UPHS (very-optimistic): & \$17/MWh \\
\end{tabular}

\subsection{LCOE comparisons for UPHS vs Li-ion Batteries}
Based on the section above (see \textit{Comparison of researched LCOE for PHS, UPHS, and Li-ion Batteries}), we will now make direct comparisons of the 40-year LCOE predictions for UPHS and Li-ion batteries. We will calculate these based on various combinations of predicted outcomes. We note that even under the best case scenario for Li-ion, UPHS is still cheaper.

\noindent\textbf{Unbiased Cases for Li-ion and UPHS}
\begin{itemize}
  \item Optimistic UPHS is 3.5 times cheaper than optimistic Li-ion (220/63)
  \item Conservative UPHS is 1.6 times cheaper than conservative Li-ion (285/177)
\end{itemize}

\noindent\textbf{Best Cases for Li-ion}
\begin{itemize}
  \item Conservative UPHS is still 1.2 times cheaper than optimistic Li-ion (220/177)
\end{itemize}

\noindent\textbf{Best Cases for UPHS}
\begin{itemize}
  \item Optimistic UPHS is 4.5 times cheaper than conservative Li-ion (285/63)
  \item Highly-optimistic UPHS is $\sim{13}$ times cheaper than optimistic Li-ion (220/17)
  \item Highly-optimistic UPHS is $\sim{17}$ times cheaper than conservative Li-ion (285/17)
\end{itemize}


